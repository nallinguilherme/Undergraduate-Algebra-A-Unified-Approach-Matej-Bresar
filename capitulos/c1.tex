\chapter{Glossary of Basic Algebraic Structures}

\section{Binary Operations}

\begin{exercicio}{1.17}
	Let \(\mathcal{P}(x)\) be the power set of the nonempty set X. The union, intersection,
	and set difference are binary operations on P(X). Determine which among them
	are associative, commutative, and have a (left or right) identity element.
\end{exercicio}


\begin{solucao}
Sejam \(A, B, C \in \mathcal{P}(X)\), assim, para a operação de união desses conjuntos, temos:

\[
\begin{array}{rcl}
	(A \cup B) \cup C & = & A \cup (B \cup C) \\
	(A \cup B) \cup C & = & A \cup (C \cup B) \\
	(A \cup B) \cup C & = & (A \cup C) \cup B
\end{array}
\]

O que mostra que a operação de união é associativa. Além disso, temos que: \(A \cup B = B \cup A\), o que mostra que a operação de união é comutativa. Por fim, o conjunto vazio é o elemento identidade para a operação de união, pois \(A \cup \emptyset = \emptyset \cup A = A\).

Para a operação de interseção, podemos verificar a associatividade:

\[
\begin{array}{rcl}
	(A \cap B) \cap C & = & A \cap (B \cap C) \\
	(A \cap B) \cap C & = & A \cap (C \cap B) \\
	(A \cap B) \cap C & = & (A \cap C) \cap B
\end{array}
\]

E o conjunto $X$ é o elemento identidade para a interseção, uma vez que: \(X \cap A = A \cap X = A\).

Por fim, a operação de diferença de conjuntos não é comutativa, tome o contraexemplo em que \(A = \{1\}\), \(B = \{1, 2\}\) e \(C = \{2\}\), assim: \(A - B = \emptyset\) mas \(B - A = {2}\). Para a associatividade, temos: \((A - B) - C = A - (B - C)\), e o conjunto vazio é o elemento identidade para a diferença de conjuntos, pois: \(A - \emptyset = A\).
\end{solucao}

\begin{exercicio}{1.18}
Determine which of the following binary operations on \(\mathbb{N}\) are associative, have a (left or right) identity element, and contain two different elements that commute:

\begin{itemize}
	\item[(a)] \(m * n = m + 2n\);
	\item[(b)] \(m * n = m^2 n\);
	\item[(c)] \(m * n = m\);
	\item[(d)] \(m * n = m^n\).
\end{itemize}
\end{exercicio}
\begin{solucao}
Sejam \(m,n,p \in \mathbb{N}\) então para o caso (a), a comutatividade é falsa, uma vez que \(m * n = m + 2n \neq n + m = n + 2m\), já a associatividade falha em:
\begin{align*}
		(m * n) * p = (m + 2n) * p = m + 2n + 2p\\
		\neq m * (n * p) = m * (n + 2p) = m + 2n + 4p
\end{align*}

Vamos verificar se há elemento neutro da operação, para isso tome $e \in \mathbb{N}$, para $e = 1$:
\begin{align*}
	&m * 1 = m + 2 \cdot 1 = m + 2 \\
	&\neq 1 * m = 1 + 2m
\end{align*}

Se considerarmos $0 \not \in \mathbb{N}$, acabou. Caso contrário, para $e = 0$:
\begin{align*}
	&m * 0 = m + 2 \cdot 0 = m \\
	&\neq 0 * m = 0 + 2m = 2m
\end{align*}

Assim, a operação admite elemento identidade à direita.

Para o item \textit{(b)}, vamos verificar a comutatividade: $m * n = m^2 n \neq n * m = n^2 m$, basta tomar $m = 2, \, n = 3$ que falha. Para a associatividade falha também, basta tomar $m = 2, n = 1, m = 3$. Seja $e \in \mathbb{N}$ o elemento neutro, então $e * m = e^2 m = em = m$, no entanto $m * e = m^2 e = m^2$, o que demonstra que a operação admite identidade à esquerda. 

Para o item \textit{(c)}, vamos verificar a comutatividade: $m * n = m \neq n * m = n$, basta que $m \neq n$ para que falhe a propriedade. Vamos verificar a associatividade:
\begin{align*}
	&m * (n * p) = m * n = m \\
	& = (m * n) * p = m * p = m \\
\end{align*}

E a existência do elemento identidade segue por: $m * e = m \neq e * m = e$, porém isso implica que, sendo $e_1 = 2, e_2 = 3 \in \mathbb{N}$ temos $m * 2 = m = m * 3 \implies 2 = 3$, o que é falso, contradizendo o proposição de que, uma vez que existe elemento neutro, ele é único.

Para \textit{(d)}, vamos verificar a comutatividade: \(m * n = m^n \neq n * m = n^m\), basta tomar \(m = 2, \, n = 3\) que falha. Para a associatividade falha também, basta tomar \(m = 2, n = 1, p = 3\). Seja \(e \in \mathbb{N}\) o elemento neutro, então \(e * m = e^m = m\), no entanto \(m * e = m^e = m\), o que demonstra que a operação admite identidade à esquerda.(d)
\end{solucao}

\begin{exercicio}{1.19}
	Let \( S \) be a set with a binary operation \( * \). If subsets \( T \) and \( T' \) of \( S \) are closed under \( * \), then so is their intersection \( T \cap T' \). Find an example showing that this does not always hold for the union \( T \cup T' \).
\end{exercicio}

\begin{solucao}
	Sejam $S = \mathbb{Z}, T = \{ 1, 0\}, T' = \{-1\}$ e a operação de adição definida em [1.1]. Assim, $1 + (-1) = 0 \not\in T \cap T'$, porém  $0 \in T \cup T'$. 
	
	\begin{flushright}
		\textit{... Procurarei um exemplo melhor.}
	\end{flushright}
\end{solucao}

\begin{exercicio}{1.20}
	Find all finite nonempty subsets of Z that are closed under addition.
\end{exercicio}

\begin{solucao}
	Seja \( A \subseteq \mathbb{Z} \) um subconjunto finito não vazio fechado sob adição. Vamos analisar os possíveis casos:

	1. Se \( A \) contém apenas um elemento, então \( A = \{a\} \) onde \( a \in \mathbb{Z} \). Neste caso, \( A \) é fechado sob adição, pois \( a + a = 2a \in A \) se e somente se \( a = 0 \). Portanto, \( A = \{0\} \).

	2. Se \( A \) contém mais de um elemento, seja \( a \) o menor elemento positivo em \( A \). Como \( A \) é fechado sob adição, todos os múltiplos de \( a \) devem estar em \( A \). No entanto, como \( A \) é finito, isso implica que \( A \) deve ser da forma \( \{0, a, 2a, \ldots, ka\} \) para algum \( k \in \mathbb{N} \).

	3. Se \( A \) contém elementos negativos, seja \( -b \) o menor elemento negativo em \( A \). Similarmente, todos os múltiplos de \( -b \) devem estar em \( A \). Portanto, \( A \) deve ser da forma \( \{-kb, \ldots, -b, 0, b, \ldots, jb\} \) para alguns \( k, j \in \mathbb{N} \).

	Portanto, os únicos subconjuntos finitos não vazios de \( \mathbb{Z} \) que são fechados sob adição são da forma \( \{0\} \) ou \( \{-ka, \ldots, -a, 0, a, \ldots, ja\} \) para algum \( a \in \mathbb{Z} \) e \( k, j \in \mathbb{N} \).
\end{solucao}

\begin{exercicio}{1.20}
	Let \( T \) be a nonempty subset of \( \mathbb{Z} \). Consider the following conditions:
	\begin{itemize}
		\item[(a)] \( T \) is closed under subtraction.
		\item[(b)] \( T \) is closed under addition.
		\item[(c)] \( T \) is closed under multiplication.
	\end{itemize}

	Show that (a) implies both (b) and (c), whereas (b) and (c) are independent conditions
	(to establish the latter, find an example of a set satisfying (b) but not (c), and an example of a set satisfying (c) but not (b)). Also, find some examples of sets satisfying (a). Can you find them all?
\end{exercicio}

\begin{solucao}
	Vamos mostrar que $(a) \implies (b)$. Tome \(a \in \mathbb{Z}\), \newline \(T = \{-ka, \ldots, -a, 0, a, \ldots, ja\}\) para algum \(a \in \mathbb{Z} \) e \( k, j \in \mathbb{N}  \), sem perda de generalidade. Sendo assim, tome $-ka$ e $ja$ e faça $(-ka) + (ja) = (-k + j)a = (j - k)a \in T$, pois $T$ é fechado por subtração.

	Vamos mostrar $(a) \implies (c)$. \textit{finalizar...}
\end{solucao}

\begin{exercicio}{1.22}
	How many binary operations are there on a set with n elements?
\end{exercicio}

\begin{solucao}
	Sejam \(A, B\) dois conjuntos finitos tais que \(|A| = n, \, |B| = m\), assim, o número de mapeamentos (ou seriam funções ?) de \(A\) para \(B\) é igual a \(m^n\), pois, para cada n-ésimo elemento de $A$ existem $m$ possibilidades de mapeamento em $B$. Este é um caso mais geral. Para a operação binária, seja $S$ um conjunto finito tal que \(|S| = n\), assim, os mapeamentos tais que \(S \times S \longrightarrow S\) são \(n^{n \cdot n} = n^{n^2}\). Pois, para cada elemento n-ésimo do conjunto de chegada $S$ temos $n \cdot n$ elementos para serem escolhidos na tupla de partida que constitui $S \times S$.
\end{solucao}

\begin{exercicio}{1.23}
	Let S be a set with two elements. Find a binary operation  on S such that \((x * y) * z \neq x * (y * z)\) for all \(x, y,z \in S\).
\end{exercicio}

\begin{solucao}
	\textit{Comentário: } Note que o enunciado diz que \(S\) tem apenas dois elementos, porém em seguida ele declara três elementos $x,y,z$ que pertencem a $S$. Caso forem distintos, há contradição. Caso contrário, $y = z$.

	Sejam $x, y, z \in S$ e a operação binária $x * y = x + 2y$. Basta tomar \(x = 1, \, y = 2, z = 1\) como exemplo.
\end{solucao}

\begin{exercicio}{1.24}
	Seja \( f \in \text{Map}(\mathbb{Z}) \) dada por \( f(n) = n + 1 \) para todo \( n \in \mathbb{Z} \). Queremos determinar todos os elementos \( g \in \text{Map}(\mathbb{Z}) \) que comutam com \( f \), ou seja, tais que:
\end{exercicio}

\begin{solucao}
	Seja \(g \in Map(\mathbb{Z})\) tal que \(g(n) = k + n, \, \forall \, k \in \mathbb{Z}\). Assim:

	\[
		(f \circ g)(n) = k + n + 1 = (g \circ f)(n) = k + n + 1
	\]
\end{solucao}

\begin{exercicio}{1.25}
	Denote by \( \text{Inj}(X) \) the set of all injective maps in \( \text{Map}(X) \), by \( \text{Sur}(X) \) the set of all surjective maps in \( \text{Map}(X) \), and by \( \text{Bij}(X) \) the set of all bijective maps in \( \text{Map}(X) \).

	\begin{enumerate}
		\item[(a)] Prove that the sets \( \text{Inj}(X) \), \( \text{Sur}(X) \), \( \text{Bij}(X) \), \( \text{Map}(X) \setminus \text{Inj}(X) \), and \( \text{Map}(X) \setminus \text{Sur}(X) \) are closed under \( \circ \).
		\item[(b)] Prove that \( \text{Inj}(X) = \text{Sur}(X) = \text{Bij}(X) \) if \( X \) is a finite set.
		\item[(c)] Prove that the set \( \text{Map}(X) \setminus \text{Bij}(X) \) is closed under \( \circ \) if and only if \( X \) is a finite set.
	\end{enumerate}
\end{exercicio}

\begin{solucao}

\end{solucao}

\section{Semigrupos e Monóides}

\begin{exercicio}{1.40}
	Show that \(\mathbb{N}\) is a semigroup under the operation \(\newline m * n = max\{m, n\}\), as well as under the operation \(m  * n = min\{m, n\}\). Which one among them is a monoid?
\end{exercicio}

\begin{solucao}
	Primeiro, seja $*$ a operação tal que, dados \(m, n, p \in \mathbb{N}\), então \(m * n = max\{m, n\}\). Assim, 
	
	\[m * (n * p) = m * \max(n, p) = \max(m, n, p) = \max\{m, n\} * p = (m*n)*p\]

	Para a operação \(m*n = \min(m,n)\):
	\begin{align*}
		&m * (n*p) = m * \min(n,p) = \min(m,n,p) \\
		& = \min(m,n) * p = (m * n) * p \\.
	\end{align*}
	Note que apenas no primeiro caso, podemos considerar \((\mathbb{N}, *)\) um monóide, uma vez que possui o elemento identidade $1 \in \mathbb{N}$:
	\begin{align*}
		m * 1 = \max(1,n) = \max(n, 1) = 1*m, \, \forall m \in \mathbb{N}
	\end{align*}
\end{solucao}

\begin{exercicio}{1.41}
	Find all binary operations on the set \(S=\{e, a\}\) for which $S$ is a semigroup and $e$ is a left identity element.
\end{exercicio}

\begin{solucao}
	Sendo $e$ identidade à esquerda, podemos definir a operação $a * e = a$.

	Para que $S$ seja um semigrupo, a operação deve ser associativa. Vamos analisar todas as possíveis operações binárias em $S = \{e, a\}$:

	1. \( e * e = e \)
	2. \( e * a = a \)
	3. \( a * e = a \)
	4. \( a * a = x \), onde \( x \in \{e, a\} \)

	Vamos verificar a associatividade para cada caso de \( a * a \):

	Caso 1: \( a * a = e \)
	\[
	(a * a) * a = e * a = a \quad \text{e} \quad a * (a * a) = a * e = a
	\]
	Neste caso, a operação é associativa.

	Caso 2: \( a * a = a \)
	\[
	(a * a) * a = a * a = a \quad \text{e} \quad a * (a * a) = a * a = a
	\]
	Neste caso, a operação também é associativa.

	Portanto, as operações binárias que fazem de \( S \) um semigrupo com \( e \) como identidade à esquerda são:

	1. \( e * e = e \), \( e * a = a \), \( a * e = a \), \( a * a = e \)
	2. \( e * e = e \), \( e * a = a \), \( a * e = a \), \( a * a = a \)
\end{solucao}

\begin{exercicio}{1.42}
	Show that $\mathbb{R}$ is a semigroup under the operation $x * y = |x|y$. Does it have a left or right identity element? For each $x \in \mathbb{R}$, find all elements commuting with $x$.
\end{exercicio}

\begin{solucao}
	Devemos verificar a associatividade em \((\mathbb{R}, *)\). Para isso, tome $x, y, z \in \mathbb{R}$, daí:

	\begin{align*}
		&x * (y * z) = x * |y|z = |x|y||z = |xy|z \\
		& =||x|y|z = |x|y * z = (x * y) * z \\
		& \implies x * (y * z) = (x * y) * z
	\end{align*}

	Agora, vamos determinar os elementos que comutam com \(x\). Para $x \geq  0$, os elementos que comutam são $p \in \mathbb{R}$ tal que $p \geq 0$ e o contrário é válido.
\end{solucao}

\begin{exercicio}{1.43}
	Show that $\mathbb{Z}$ is a monoid under the operation $m * n = m + n + mn$. Find all
	its invertible elements.
\end{exercicio}

\begin{solucao}
	Vamos mostrar que $(\mathbb{Z}, *)$ é um monóide. Verifiquemos a associatividade, para isso, considere \(a, b, c \in \mathbb{Z}\).
	\begin{gather*}
		a * (b * c) = a * (b + c + bc) = a + (b + c + bc) + ab + ac + abc \\ 
		(a*b)*c = (a + b + ab) * c = (a + b + ab) + c + ac + bc + abc \\
		\therefore a * (b * c) = (a*b)*c.
	\end{gather*}

	Note que o elemento identidade da operação definida é \(0 \in \Z\): \(a * 0 = a + 0 + a0 = a = 0 + a + 0a = 0 * a\) e, pela sua existência, $(\mathbb{Z}, *)$ é monóide.

	O único elemento invertível do monóide é \(0 \in \Z\).
\end{solucao}

\begin{exercicio}{1.44}
	Prove that in each monoid with identity element $e$,$x * y * x = e$ implies that $x$ and $y$ are invertible and $y = x^{-2}$.
\end{exercicio}

\begin{solucao}
	Primeiro, vamos provar que $x*y*x = e \implies \exists \, \, x^{-1}, y^{-1}$. Observe que \(y * x\) é inversa a direita de $x$ e \(x * y\) é inversa a esquerda de \(x\), assim, \(x\) é invertível. Para \(y\) fazemos: \(x * y * x * (y * x) = y * x \implies x * y = y * x\) o que mostra que existe $y^{-1}$.

	Segunod, vamos provar que $y = x^{-2}$.
\end{solucao}